\chapter{Correção}

No capítulo anterior vimos que é possível utilizar o método empírico para avaliar tempo de processamento de diferentes soluções para um mesmo problema.
Neste capítulo daremos um passo atrás.
Como podemos garantir, para começo de conversa, que um algoritmo de fato resolve um problema?
Em outras palavras, como podemos provar a correção de um algoritmo?

Para provar a correção de um algoritmo iremos usar uma forma de prova por indução.
Assim, antes de partir para os exemplos de prova por indução em um algoritmo, vale a pena relembrar como funciona uma prova por indução em um contexto mais típico.

Normalmente, uma prova por indução é usada para provar alguma propriedade sobre números naturais.
A ideia de uma prova por indução é relativamente simples.
Ela se divide em três etapas.
Primeiro precisamos provar que a propriedade vale para o $0$ ou para o primeiro número que nos interessa.
Isso é chamado de {\em Base da Indução}.
Então supomos que a propriedade vale para um número $n$, {\em Hipótese da Indução}.
Por fim, provamos que se vale para $n$ então vale para $n+1$, {\em Passo de Indução}.
Assim, mostramos que vale para $0$ e se vale para $0$, deve valer para $1$, e se vale para $1$, deve valer para $2$ e assim por diante.
Com isso, provamos que a propriedade vale para todos os números naturais.

Vejamos um exemplo.
Considere a seguinte somatória:

\begin{displaymath}
  1 + 2 + 3 + \dots + n = \sum_{i=1}^n i = \frac{n(n+1)}{2}
\end{displaymath}

Vamos provar por indução que esse resultado vale para qualquer número natural $n$.

O primeiro passo é provar a base da indução.
Vamos provar que o resultado vale para $n = 1$

\begin{displaymath}
  1 = \frac{1(1+1)}{2}
\end{displaymath}

Agora vamos explicitar a Hipótese de Induação.

\begin{displaymath}
  1 + 2 + 3 + \dots + n = \frac{n(n+1)}{2}
\end{displaymath}

Por fim, fazemos o Passo de Indução.

\begin{eqnarray*}
  1 + 2 + 3 + \dots + n + n + 1 & = & \frac{n(n+1)}{2} + n + 1 \\
  & = & \frac{n^2 + n + 2n + 2}{2} \\
  & = & \frac{n^2 + 3n + 2}{2} \\
  & = & \frac{(n+1)(n+2)}{2} \\
\end{eqnarray*}

Note que a primeira equação vale por conta da Hipótese de Indução.

Passemos agora para um problema computacional.
Continuemos considerando o problema da busca em uma sequência ordenada e os dois algoritmos que conhecemos para ele.
Primeiro o algoritmo da busca sequencial:

\begin{codebox}
\Procname{$\proc{BuscaSequencial}(A, b)$}
\li \For $i \gets 1$ até $n$
\li \Do \If $a_i = b$
\li     \Then \Return $i$
        \End
    \End
\li \Return $\bot$
\End
\end{codebox}

Para mostrar que o algoritmo é correto temos que incontrar um {\em invariante}.
Uma propriedade que vale em todas as iterações.
No nosso caso, vamos considerar a linha 2 do algoritmo e a propriedade será a seguinte:

\begin{displaymath}
b \textrm{ não ocorre em } a_1, \dots, a_{i-1}
\end{displaymath}

Vamos provar que essa propriedade é de fato invariante usando a técnica da indução.

Primeiro temos que provar que ela vale para $i = 1$.
Essa é a base da indução.

Para isso basta notar que neste caso a sequência é vazia e, portanto, $b$ não pertence a ela.

A Hipótese de Indução já foi explicitada.
Para mostrar o passo de indução vamos supor a HI e mostrar que a propriedade continua valendo para $i+1$.
Se $b$ estivesse na sequência $a_1 \dots a_i$ então, pela HI, $b = a_i$.
Neste caso, não chegaríamos na linha 2, porque o algoritmo teria encerrado antes disso.

Assim, sempre que chegamos na linha 2, $b$ não ocorre em $a_1, dots, a_{i-1}$.
Portanto, quando chegamos na 3, é a primeira vez em que $a_i = b$.
E se chegarmos na linha 4, sabemos que $b$ não ocorre em $a_1, \dots, a_n$.


Vamos agora para nosso segundo exemplo:


\begin{codebox}
  \Procname{$\proc{BuscaBinaria}(A, b)$}
  \li $i \gets 1$
  \li $j \gets |A|$
  \li \While $i \leq j$
  \li \Do $m \gets \left \lfloor{\frac{j+i}{2}}\right\rfloor$
  \li \If $b < a_m$
  \li     \Then $j \gets m - 1$
  \li \Else
      \If $b > a_m$
  \li      \Then $i \gets m + 1$
  \li \Else \Return m 
      \End
  \End
  \End
  \li \Return $\bot$
\end{codebox}

Vamos mostrar que as seguintes propriedades são invariantes na linha 4:

\begin{displaymath}
b \textrm{ não ocorre em } a_1, \dots, a_{i-1}
\end{displaymath}

\begin{displaymath}
b \textrm{ não ocorre em } a_{j+1}, \dots, a_n
\end{displaymath}

A base da indução é simples, no primeiro momento $i = 1$ e $j = n$.
Portanto ambas propriedades valem porque as duas sequências são vazias nessas condições.

Vamos supor que a propriedade vale em um certo momento quando chegamos na linha 4.
Agora imagine que chegamos mais uma vez nessa linha.
Neste caso, não saímos do laço.
Portanto, uma de duas coisas teve que ocorrer:
$b < a_m$ ou $b >a_m$.

No primeiro caso, temos que $j = m-1$.
Como a sequência está ordenada, $b$ não ocorre em $a_{j+1} = a_m, \dots, a_n$ porque $b < a_m$.
Além disso, pela hipótese de indução, temos que $b$ não ocorre em $a_1, \dots, a_{i-1}$.

No segundo caso, temos que $i = m+1$.
Como a sequência está ordenada, $b$ não ocorre em $a_{1} = a_{1}, \dots, a_n$ porque $b > a_m$.
Além disso, pela hipótese de indução, temos que $b$ não ocorre em $a_{j+1}, \dots, a_n$.

O invariante vale sempre na linha 4.
Se chegarmos na linha 9 é proque $a_m = b$ e se chegarmos na linha 10 é porque $b$ não está nem em $a_1, \dots, a_i$ nem em $a_j, \dots, a_n$ e $i > j$.
Portanto $b$ não está em $a_1, \dots, a_n$.

\vspace{2cm}

\begin{exercicio}
  Considere o seguinte algoritmo:

\begin{codebox}
  \Procname{$\proc{3Soma}(A, B, C)$}
  \li $n \gets 0$
  \li \For $i \gets 1$ até $n$
  \li \Do \For $j \gets 1$ até $n$
  \li     \Do \For $k \gets 1$ até $n$
  \li         \If $a_i + b_j + c_k = 0$
  \li         \Then $n \gets n + 1$
  \End
  \End
  \End
  \li \Return $n$
\end{codebox}

Prove que este algoritmo resolve o problema da 3-soma apresentado no Capítulo \ref{cha:intro}.
  
\end{exercicio}
