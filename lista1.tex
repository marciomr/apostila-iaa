\documentclass[a4,12pt]{article}
\usepackage[portuguese]{babel}
\usepackage[utf8]{inputenc}
\usepackage[portugues]{mystyle}
\usepackage{clrscode}


\newcounter{mycounter}
\setcounter{mycounter}{0}
\newenvironment{exercicio}{\refstepcounter{mycounter}
    {\bf Exercício~\themycounter: }
      \rmfamily}{\medskip}

\begin{document}

\author{Márcio Moretto Ribeiro}

\title{Lista 1: Introdução à Análise de Algoritmos}

\maketitle

  {\bf Problema da 3-soma}\\

  {\bf Entrada:} Três sequência de $n \in \mathbb{N}$ valores cada $ a_1, \dots, a_n$,  $b_1, \dots, b_n$ e $c_1, \dots, c_n$ em que $a_i, b_i, c_i \in \mathbb{Z}$ para $1 \leq i \leq n$.\\

  {\bf Saída:} A quantidade de $i$s, $j$s e $k$s tais que $a_i + b_j + c_j = 0$.

  \vspace{2cm}
  
\begin{exercicio}

  Considere o seguinte algoritmo:

\begin{codebox}
  \Procname{$\proc{3Soma}(A, B, C)$}
  \li $m \gets 0$
  \li \For $i \gets 1$ até $n$
  \li \Do \For $j \gets 1$ até $n$
  \li     \Do \For $k \gets 1$ até $n$
  \li         \If $a_i + b_j + c_k = 0$
  \li         \Then $m \gets m + 1$
  \End
  \End
  \End
  \li \Return $m$
\end{codebox}

Prove que este algoritmo resolve o problema da 3-soma, ou seja, que ele é correto.
  
\end{exercicio}

\newpage

\begin{exercicio}
Considere o seguinte algoritmo $3Soma$ apresnetado no final do capítulo anterior.
Calcule o tempo de processamento em função do tamanho $n$ da entrada assumindo que:

\begin{itemize}
\item Cada iteração de variável toma tempo constante $c_1$
\item Cada atribuição toma tempo constante $c_2$
\item Cada soma toma tempo constante $c_3$
\item A saída toma tempo constante $c_4$
\end{itemize}
  
\end{exercicio}


\begin{exercicio}
  Mostre que o tempo de processamento do algorítimo $3Soma$ é $\Theta(n^3)$ no pior caso.
\end{exercicio}


\begin{exercicio}
  Descreva em pseudo-código um algoritmo cujo tempo de processamento no pior caso é $\Theta(n^2log(n))$.
  Dica: ordene a sequência $c_1, \dots, c_n$ usando qualquer um dos métodos visto em aula e use a busca binária.
\end{exercicio}

\end{document}
